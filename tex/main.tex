\documentclass[12pt,a4paper]{article}
\usepackage[utf8]{inputenc}
\usepackage[brazil]{babel}
\usepackage[top=3cm,left=3cm,right=3cm]{geometry}
\usepackage{graphicx}
\usepackage{import}
\usepackage{enumitem}
\usepackage{xcolor}
\usepackage{ulem}
\usepackage{color}


\renewcommand{\baselinestretch}{1.3}

\newcommand{\calico}{\textbf{CALICO}}

\newcommand{\destaque}[1]{\uline{#1}}

\newcommand{\parun}{\textbf{Parágrafo único.}\ }

\newcommand{\hilight}[1]{\colorbox{yellow}{#1}}

\title{Estatuto}
\author{Centro Acadêmico Livre De Computação}
\date{\today}

\begin{document}

\begin{titlepage}
	\centering
	{\scshape\Large Universidade Federal de Santa Catarina\par}
    \vspace{1cm}
	{\scshape\huge Centro Acadêmico Livre da Computação \par}
	
	\vspace{2cm}
	
	\includegraphics[width=\textwidth]{calico.png}
	
	\vspace{2cm}
	{\huge\bfseries Estatuto\par}

	\vfill

	{\large \today\par}
\end{titlepage}


\begin{enumerate}[label=\textbf{Art. \arabic*º.}]

\section{DISPOSIÇÕES PRELIMINARES}

\subsection{Natureza, Princípios e Objetivos}

    \item O \textbf{CENTRO ACADÊMICO LIVRE DA COMPUTAÇÃO}, doravante denominado \calico\ é a entidade representativa das e dos estudantes do Curso de Graduação em Ciência da Computação da Universidade Federal de Santa Catarina e rege-se pelo presente estatuto.
    
    \item São princípios básicos do \calico:
    \begin{enumerate}[label=\textbf{\Roman* - }]
        \item a defesa dos interesses das e dos estudantes do Curso de Ciência da Computação da UFSC;
        \item a defesa da universidade pública, gratuita e de excelência;
        \item o aperfeiçoamento das atividades acadêmicas do Curso.
    \end{enumerate}

    \item São objetivos do \calico, dentre outros:
    \begin{enumerate}[label=\textbf{\Roman* - }]
        \item a promoção da integração:
            \begin{enumerate}[label=\textbf{\alph*)}]
                \item entre seus membros;
                \item entre os segmentos discente e docente do curso;
                \item com os outros setores da universidade;
                \item com as demais entidades estudantis;
                \item com todos os setores da comunidade.
            \end{enumerate}
        \item promover semestralmente a recepção e integração dos calouros;
        \item manter um veículo de comunicação periódica com os estudantes do Curso;
        \item representar as e os estudantes do Curso nos órgãos colegiados da UFSC;
        \item promover cursos, palestras e outras atividades sociais e acadêmicas.
    \end{enumerate}

\section{COMPOSIÇÃO}
    
\subsection{Membros}
    
    \item São membros do \calico\ todas e todos os estudantes matriculados no Curso de Graduação em Ciência da Computação da Universidade Federal de Santa Catarina.
    
    \item São direitos dos membros:
    \begin{enumerate}[label=\textbf{\Roman* - }]
        \item ser informado e participar das atividades do \calico;
        \item ser ouvido e respeitado em suas opiniões, propostas e posições;
        \item votar e ser votado nos termos deste estatuto.
    \end{enumerate}
    
    \item São deveres dos membros:
    \begin{enumerate}[label=\textbf{\Roman* - }]
        \item respeitar e cumprir os preceitos estipulados por este estatuto e as decisões tomadas pelos órgãos estatutários;
        \item zelar pelo patrimônio da entidade e auxiliar na sua manutenção;
        \item quando investido de qualquer cargo do \calico, cumprir com dedicação e responsabilidade e agir com base nos princípios da legalidade e transparência.
    \end{enumerate}

\subsection{Patrimônio}
    
    \item Constituem patrimônio do \calico\ todos os bens, materiais e imateriais que a entidade possua ou venha a adquirir.
    
    \item Para efetuar a alienação de qualquer bem, a Diretoria deverá obter autorização em reunião aberta a todos os membros.
    
    \item Extinguindo-se o \calico, sem que haja outra entidade representativa dos estudantes de Ciência da Computação da UFSC, todo o patrimônio da entidade será transferido para o Conselho de Entidades Estudantis do Centro Tecnológico (CETEC).
    
\section{ORGANIZAÇÃO}
    
\subsection{Estrutura}

    \item O \calico\ é composto por dois órgãos, a Assembleia Geral e a Diretoria.
    
\subsection{Assembleia Geral}

    \item A Assembleia Geral é a instância deliberativa máxima do \calico\ e constitui-se na reunião de todos os seus membros.

    \item A Assembleia Geral reúne-se extraordinariamente em caso de necessidade relevante, podendo ser convocada:

    \begin{enumerate}[label=\textbf{\Roman* - }]
        \item pela Diretoria;
        \item por 10\% dos membros do \calico.
    \end{enumerate}

    \parun Em qualquer uma dessas hipóteses, a Assembleia Geral deverá ser convocada por edital que defina a pauta e divulgada por meio de comunicação fornecido pela universidade e com capacidade de atingir todos os estudantes do curso, com pelo menos 48 horas de antecedência.

    \item A Assembleia Geral será presidida por uma mesa eleita no início dos trabalhos.
    
     \begin{enumerate}[label=\textbf{\S\arabic*º.}]
        \item{À mesa compete, dentre outras atividades, organizar a lista de presença, realizar o controle das votações, redigir a ata e divulgá-la.}
        \item{É garantida uma vaga na mesa para um membro da Diretoria, exceto no caso de a Assembleia Geral ter sido convocada com o fim de destituir a Diretoria ou em caso de inexistência desta.}
     \end{enumerate}
    
    \item Compete à Assembleia Geral:
    \begin{enumerate}[label=\textbf{\Roman* - }]
        \item deliberar sobre assuntos de alta relevância para o Centro Acadêmico ou sobre quaisquer outros assuntos que a ela venha a se encaminhar;
        \item alterar o Estatuto;
        \item interpretar, em última instância, o Estatuto e resolver casos omissos;
        \item destituir parcial ou totalmente a Diretoria e indicar comissão provisória da gestão.
    \end{enumerate}

    \item O quórum da Assembleia Geral é de 10\% dos membros do \calico\ em primeira chamada, e de 5\% em segunda chamada, após quinze minutos.
    
    \parun As decisões serão tomadas por maioria simples dos presentes, mediante prévia deliberação, salvo no caso de destituição parcial ou total da gestão, onde também será necessário aprovação de mais de 15\% dos membros do curso.
    
    \item Para alteração do Estatuto a Assembleia deverá ser convocada tendo tal objetivo como pauta única.
    
    \item Em caso de destituição total da gestão a Assembleia Geral elegerá a Comissão Eleitoral, que marcará eleições extraordinárias, as quais se realizarão no prazo máximo de 21 dias.
    \begin{enumerate}[label=\textbf{\S\arabic*º.}]
        \item{Aplica-se, no que couber, o disposto na seção "Eleições para a Diretoria".}
        \item{A Comissão Eleitoral é eleita como Diretoria provisória até a posse da nova gestão.}
        \item{O mandato da nova Diretoria tem prazo final no mesmo dia em que terminaria o da Diretoria destituída, salvo se restarem 60 dias ou menos para o término da gestão, hipótese na qual às eleições subsequentes serão antecipadas.}
     \end{enumerate}
    

    
\subsection{Diretoria}

    \item A Diretoria é a equipe que dirige a entidade, sendo eleita anualmente e composta originalmente por no mínimo cinco membros.

    \item A estrutura da Diretoria é livre, devendo necessariamente prever:
    \begin{enumerate}[label=\textbf{\Roman* - }]
        \item Presidente(a)
        \item Vice-presidente(a)
        \item Tesoureiro(a)
        \item Secretário(a)
    \end{enumerate}

    \item São deveres da Diretoria:
    \begin{enumerate}[label=\textbf{\Roman* - }]
        \item responder pelas ações da entidade;
        \item cumprir com os objetivos e princípios;
        \item convocar as eleições;
        \item indicar os representantes discentes aos órgãos colegiados;
        \item gerir administrativa e financeiramente a entidade.
    \end{enumerate}

    \item A presidência responde pela Diretoria em caso de controvérsia e, para todos os fins, representa legalmente a entidade.

    \item À tesouraria cabe a gerência das finanças e patrimônio, bem como a organização da prestação de contas.

    \item À secretaria cabe a redação dos documentos e atas, bem como a gerência de documentos da entidade.

\subsection{Eleições para a Diretoria}

    \item As eleições para a Diretoria se realizarão anualmente, durante o período letivo, e pelo menos três semanas antes do término do segundo semestre letivo.

    \parun Em caso de atraso nas eleições, o mandato da Diretoria será reduzido para obedecer à data estipulada neste artigo.

    \item Cabe à Diretoria convocar as eleições, nomeando a Comissão Eleitoral em reunião aberta.

    \begin{enumerate}[label=\textbf{\S\arabic*º.}]
        \item{A Comissão Eleitoral deverá ser composta por no mínimo três membros do \calico.}
        \item{Os membros da Comissão Eleitoral não poderão integrar nenhuma das chapas inscritas.}
     \end{enumerate}

    \item Compete à Comissão Eleitoral:
    \begin{enumerate}[label=\textbf{\Roman* - }]
        \item definir o prazo de inscrição de chapas e a data da eleição, respeitando:
            \begin{enumerate}[label=\textbf{\alph*)}]
                \item entre uma e duas semanas para a inscrição de chapas;
                \item entre duas e três semanas entre a homologação das inscrições e a data da eleição.
            \end{enumerate}

        \item definir a forma com que a eleição será realizada (presencial ou virtualmente);
        \item definir mesários, caso a eleição seja presencial;
        \item divulgar amplamente todas as informações relacionadas à eleição, utilizando os meios de comunicação oferecidos pela universidade e com capacidade de atingir todos os estudantes do curso;
        \item realizar debates entre as chapas inscritas, caso julgar necessário.
    \end{enumerate}

    \item Em caso de haver registro de apenas uma chapa, a Comissão Eleitoral deve divulgar o ocorrido e estender o período de inscrição de chapas por uma semana, tendo em vista que esse período de prorrogação será concomitante ao período compreendido pelo item I.b) do Art. 26º.

    
\section{Disposições Finais}

    \item O presente estatuto entra em vigor na data de sua aprovação em Assembleia Geral e revoga todas as disposições anteriores.

\end{enumerate}

\end{document}

% §